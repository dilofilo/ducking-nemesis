\documentclass[]{article}
\usepackage{sectsty}
\usepackage[parfill]{parskip}
\usepackage{hyperref}
\sectionfont{\fontsize{10}{10}\selectfont}

\begin{document}

\title{COP 290 Assignment 1}
\author{Kabir Chhabra , Harman Kumar , Haroun Habeeb}
\maketitle

\begin{abstract}
\begin{flushleft}
Problem Statement : \href { http://www.cse.iitd.ac.in/~prathmesh/ta/COP290/ass1.html }{Assignment Link}
\end{flushleft}

\begin{flushleft}
The project was written in C++ using OpenGL, pthreads, mutex locks and it is aimed at ubuntu operating system. The collisions between the balls were considered to be perfectly elastic and the table was considered to be frictionless.
\end{flushleft}

\end{abstract}

\section{Overall Design}
\begin{flushleft}
In the project three classes have been used namely Ball, Table and Screensaver. The user enters the number of balls and the velocity corresponding to the balls can be increased or decreased by the user. Every ball runs on a thread and collisions between two balls and between the balls and the boundary are handled.
\end{flushleft} 

\section{Sub Components}
\begin{flushleft}
The Class Ball has data members for storing the X and Y co-ordinates of it's centre, radius of the ball, X and Y components of it's velocity and the mass of the ball is assumed to be proportional to the cube of it's radius. 
\\
The member functions of the class Ball include the set and get functions that are used to access the above mentioned data members and modify their values respectively, a function to display the ball and a function to modify the location in accordance with the screen size if it's changed.
\\ 
When an object of type ball is created, it is randomly assigned a centre, radius, velocity.
\end{flushleft}

\begin{flushleft}
The class Table has data members corresponding to the coordinates of the vertices and a vector for storing the colour of the table. It has a member function display to display the table and a member function reshape to resize the table if the screen size is changed.  
\end{flushleft}

\begin{flushleft}
An object of the Class ScreenSaver is called when the program starts executing. It creates N (Number of balls entered at the time of execution) objects of the class Ball, N pthreads to control the motion of each ball, N mutex locks and one object of the class table. \\
The data members are as follows: \\
\begin{enumerate}
\item isPaused (Boolean) : To check whether the execution is paused or running.
\item isFullScreen (Boolean) : To check whether the screen is in fullscreen mode or not.
\item numThreads (Integer) : The number of balls the user wants on the screen.
\item windowID (Integer) : The ID of the window corresponding to the objectof the Class ScreenSaver.
\item table (Table) : The table in the background of the window.
\item balls (Array of type Ball) : The N balls that move on the table.
\item threads (vector of pthread\_t : These are the N threads that control one ball each.
\item threadUpdate (Vector of Boolean) : This controls the individualThread function.
\item vecMutex (vector of mutex locks): It's used to lock the parameters of each ball so that only one thread can access it at a certain point of time. 
\end{enumerate}

The functions of the class are the following:
\begin{enumerate}
\item void* individualThread(void*) : This function is what each of the n-pthreads executes. It is triggered after every iteration of the display() function. The purpose of this function is to update the xCentres,yCentres, velocities of the balls. To do so, it also checks for collisions.
\item void exitter() : The exitter function is called upon pressing the Escape key. It handles all the threads, joins them, destroys all the custom made objects. Essentially just memory handling.
\item void init() : The init() function handles the initialization of the glut window. It takes care of one-time statements for the GUI to run.
\item void makeObjects( args ) : The args are references to the locations where objects will be made. The objects we are talking about are the table and the balls. 
\item void reshape(int,int) : This function handles reshaping of the window.
\item void keyHandler(char,int,int) : This function handles the pressing of keys. For example, the Escape key quits the screenSaver.
\item void display() : The display function is called by glut's mainLoop . It calls the display function of the balls and the table, then proceeds to tell each individual thread to update its variables. It also posts a redisplay request.
\item void execute() : This is where the flow of the program is controlled. It calls functions in the right order, initializes variables and then starts the glutMainLoop.
\end{enumerate}

The subcomponents, seperated by functionality, are :
\begin{itemize}
\item GUI : The GUI was handled by giving the initialization part to one class, and then allowing the balls and the table to display themselves.
\item Physics : The physics are handled by the threads. The threads call a solve() function which accepts as parameters velocities, masses and co-ordinates. It then returns the same values AFTER collision.
\item Threading : The updates of the co-ordinates and velocities of balls are handled by threads. They communicate through barrier communication ( accessing static global variables ). Mutex is applied at appropirate locatons.
\end{itemize}
 
\end{flushleft}

\section{Testing}
The components that require testing are :
\begin{enumerate}
\item GUI : Because the display() functions are modular, we just need to test the display function of the ball and that of the table.
\item Physics : Since the physics is put into a function, testing the correctness of the function is easy, using dry runs.
\item Collision Handling : Perfected by eyesight and physics.
\item Memory Handling : Valgrind will be used to ensure that there are no memory leaks.
\end{enumerate}
\section{Sub-component Interaction}
\begin{flushleft}
This heading could be divided into the following sub-
\end{flushleft}

\section{Interthread Communication}
Interthread communication is done through barrier communication. Threads modify global static variables, namely, the state variables of the balls.
To ensure that this part doesn't face issues, we lock the portion of code where the thread is modifying the values.
\begin{flushleft}

\end{flushleft}
\section{Ball Speed}
\begin{flushleft}
The X and Y components of the velocity corresponding to each ball are stored in the  object of the class Ball and can lie in a certain range. When a collision between two balls or between a ball and the wall occours, these parameters are modified in accordance with the equations derived from Newton's Laws of Motion.  
\end{flushleft}


\end{document}